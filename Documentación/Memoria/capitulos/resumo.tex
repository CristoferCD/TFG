\pagestyle{plain}
\chapter*{Resumo}

La paralelización de un código busca adaptar la ejecución secuencial de forma que se ejecuten varias instrucciones al mismo tiempo. Esto no solo reduce el tiempo de ejecución si no que permite abordar problemas de mayor tamaño y aprovechar clústeres de computación con múltiples nodos. De esta forma el código se puede escalar más fácilmente y asignar un mayor número de recursos a su ejecución.\\

En este proyecto estudiaremos el módulo de resolución de problemas estocásticos de Pyomo y estudiaremos herramientas que nos permitan adaptar el algoritmo existente a una ejecución paralela.

% TODO: Ampliar algo