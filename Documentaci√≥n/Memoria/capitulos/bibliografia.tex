\markboth{BIBLIOGRAFÍA}{BIBLIOGRAFÍA}
\addcontentsline{toc}{chapter}{Bibliografía}


\begin{thebibliography}{99}
% % EXEMPLO DE DOCUMENTO DESCARGADO DA WEB
% \bibitem{cuda} Nvidia CUDA programming guide. Versión 2.0, 2010. Dispoñible en % {\it http://www.nvidia.com}.
% 
% % EXEMPLO DE PÁXINA DA WIKIPEDIA
% \bibitem{cdma} Acceso múltiple por división de código. Artigo da wikipedia (% {\it http://es.wikipedia.org}). Consultado o 2 de xaneiro do 2010.
% 
% % EXEMEPLO DE LIBRO
% \bibitem{gonzalez} R.C. Gonzalez e R.E. Woods, {\it Digital image processing}, % 3ª edición, Prentice Hall, New York, 2007.
% 
% % EXEMPLO DE ARTIGO DE REVISTA
% \bibitem{patricia} P. González, J.C. Cartex e T.F. Pelas, ``Parallel % computation of wavelet transforms using the lifting scheme'', {\it Journal of % Supercomputing}, vol. 18, no. 4, pp. 141-152, junio 2001.

\bibitem{learningSpark} Matei Zaharia, Patrick Wendell, Andy Konwinski, Holden Karau, {\it Learning Spark: Lightning-Fast Big Data Analysis}, 1ª edición, O'Reilly Media, 2015.

\bibitem{stochasticProgramming} John R. Birge, François Louveaux, {\it Introduction to Stochastic Programming}, 2ª Edición, Springer, 2011.

\bibitem{progressiveHedging}  Watson, Jean-Paul and Woodruff, David L. and Strip, David R., {\it Progressive Hedging Innovations for a Class of Stochastic Resource Allocation Problems} (September 15, 2008). UC Davis Graduate School of Management Research Paper No. 05-08. Available at SSRN: \url{https://ssrn.com/abstract=1268385} or \url{http://dx.doi.org/10.2139/ssrn.1268385}. Consultado 17 de julio 2018.

\bibitem{IEEE829} IEEE Std 829-2008, IEEE Standard for Software and System Test Documentation

\bibitem{pyomo} Pyomo \url{http://www.pyomo.org/about/}. Consultado 17 de julio 2018.

\bibitem{thesisHydro} Andrés Guillermo Iroume Awe, {\it PROGRESSIVE  HEDGING APLICADO A COORDINACION HIDROTERMICA}, Marzo 2012, Disponible en \url{http://repositorio.uchile.cl/bitstream/handle/2250/114109/cf-iroume_aa.pdf;sequence=1}. Consultado 17 de julio 2018.

\bibitem{pyro} Pyro. \url{https://pythonhosted.org/Pyro4/}. Consultado 17 de julio 2018.

\bibitem{bigdata} Big Data. Wikipedia (\url{https://en.wikipedia.org/wiki/Big\_data}). Consultado 17 de julio 2018.

\bibitem{AMPL} AMPL. Wikipedia (\url{https://en.wikipedia.org/wiki/AMPL}). Consultado 18 de julio 2018.

\bibitem{cplex} IBM ILOG CPLEX Optimization Studio. Wikipedia (\url{https://en.wikipedia.org/wiki/CPLEX}). Consultado 22 de julio 2018.

\bibitem{glpk} GNU Linear Programming Kit (GLPK). Wikipedia (\url{https://en.wikipedia.org/wiki/GNU_Linear_Programming_Kit}). Consultado 22 de julio 2018.

\bibitem{minos} Minos (\textit{Modular In-core Nonlinear Optimization System}). Wikipedia (\url{https://en.wikipedia.org/wiki/MINOS\_(optimization\_software)}). Consultado 22 de julio 2018.

\bibitem{spark} Apache Spark. \url{http://spark.apache.org/}. Consultado 22 de julio 2018.

\bibitem{mpi} \textit{Message Passing Interface}. Wikipedia (\url{https://en.wikipedia.org/wiki/Message_Passing_Interface}). Consultado 22 de julio 2018.

% Files local to the project

\bibitem{local_funcionamientoPH} Funcionamiento PH en ``/Documentación/Análisis/Funcionamiento PH paralelo.pdf''
\bibitem{local_distributedTestApp} Proyecto de pruebas en: ``/DistributedTest''
\bibitem{local_pyspdoc} Documentación de PySP en ``/pyomo/doc/pysp/pyspdoc.pdf''

\end{thebibliography}

