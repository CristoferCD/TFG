\markboth{BIBLIOGRAFÍA}{BIBLIOGRAFÍA}
\addcontentsline{toc}{chapter}{Bibliografía}


\begin{thebibliography}{99}
% % EXEMPLO DE DOCUMENTO DESCARGADO DA WEB
% \bibitem{cuda} Nvidia CUDA programming guide. Versión 2.0, 2010. Dispoñible en % {\it http://www.nvidia.com}.
% 
% % EXEMPLO DE PÁXINA DA WIKIPEDIA
% \bibitem{cdma} Acceso múltiple por división de código. Artigo da wikipedia (% {\it http://es.wikipedia.org}). Consultado o 2 de xaneiro do 2010.
% 
% % EXEMEPLO DE LIBRO
% \bibitem{gonzalez} R.C. Gonzalez e R.E. Woods, {\it Digital image processing}, % 3ª edición, Prentice Hall, New York, 2007.
% 
% % EXEMPLO DE ARTIGO DE REVISTA
% \bibitem{patricia} P. González, J.C. Cartex e T.F. Pelas, ``Parallel % computation of wavelet transforms using the lifting scheme'', {\it Journal of % Supercomputing}, vol. 18, no. 4, pp. 141-152, junio 2001.

\bibitem{learningSpark} Matei Zaharia, Patrick Wendell, Andy Konwinski, Holden Karau, {\it Learning Spark: Lightning-Fast Big Data Analysis}, 1ª edición, O'Reilly Media, 2015.

\bibitem{stochasticProgramming} John R. Birge, François Louveaux, {\it Introduction to Stochastic Programming}, 2ª Edición, Springer, 2011.

% Files local to the project

\bibitem{local_funcionamientoPH} PONER AQUÍ LA RUTA

\end{thebibliography}

