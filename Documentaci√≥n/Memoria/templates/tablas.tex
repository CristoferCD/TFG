\usepackage{tabularx}
\usepackage{xkeyval}

% TABLA TAREA EDT

% define the key (arguments)
\makeatletter
\define@key{wpkeys}{id}{%
  \def\wpid{#1}
}
\define@key{wpkeys}{name}{%
  \def\wpname{#1}
}
\define@key{wpkeys}{duration}{%
  \def\wpduration{#1 días}
}
\define@key{wpkeys}{description}{%
  \def\wpdescription{#1}
}
\define@key{wpkeys}{results}{%
  \def\wpresults{#1}
}
\makeatother
% end of key definition

% new command
\newcommand{\WorkItem}[2][]{%
    \setkeys{wpkeys}{#1}%
    \begin{table} [H]
      \begin{tabularx}{\linewidth}{|p{3cm}|X|}
        \hline
        \textbf{Id} & \wpid \tabularnewline
        \hline
        \textbf{Tarea} & \wpname \tabularnewline
        \hline
        \textbf{Duración} & \wpduration \tabularnewline
        \hline
        \textbf{Descripción} & \wpdescription \tabularnewline
        \hline
        \textbf{Salida} & \wpresults \tabularnewline
        \hline
      \end{tabularx}
      #2
    \end{table}
}
% end of command definition

% TABLA REQUISITO

% define the key (arguments)
\makeatletter
\define@key{rqkeys}{id}{%
  \def\rqid{#1}
}
\define@key{rqkeys}{name}{%
  \def\rqname{#1}
}
\define@key{rqkeys}{description}{%
  \def\rqdescription{#1}
}
\define@key{rqkeys}{priority}{%
  \def\rqpriority{#1}
}
\makeatother
% end of key definition

% new command
\newcommand{\Req}[2][]{%
    \setkeys{rqkeys}{#1}
    \begin{table}[H]
      \begin{tabularx}{\linewidth}{|p{3cm}|X|}
        \hline
        \textbf{\rqid} & \rqname \tabularnewline
        \hline
        \textbf{Descripción} & \rqdescription \tabularnewline
        \hline
        \textbf{Importancia} & \rqpriority \tabularnewline
        \hline
      \end{tabularx}
      #2
    \end{table}
}
% end of command definition

% TABLA PRUEBA

% define the key (arguments)
\makeatletter
\define@key{tkeys}{id}{%
  \def\tid{#1}
}
\define@key{tkeys}{name}{%
  \def\tname{#1}
}
\define@key{tkeys}{requirements}{%
  \def\treq{#1}
}
\define@key{tkeys}{objective}{%
  \def\tobjective{#1}
}
\define@key{tkeys}{techniques}{%
  \def\ttechniques{#1}
}
\define@key{tkeys}{testCases}{%
  \def\ttestcases{#1}
}
\define@key{tkeys}{expected}{%
  \def\texpected{#1}
}
\makeatother
% end of key definition

% new command
\newcommand{\TestItem}[2][]{%
    \setkeys{tkeys}{#1}%
    \begin{table} [H]
      \begin{tabularx}{\linewidth}{|p{3cm}|X|}
        \hline
        \textbf{Prueba   \tid} & \tname \tabularnewline
        \hline
        \textbf{Objetivo} & \tobjective \tabularnewline
        \hline
        \textbf{Requisitos validados} & \treq \tabularnewline
        \hline
        \textbf{Técnicas aplicadas} & \ttechniques \tabularnewline
        \hline
        \textbf{Casos de prueba} & \ttestcases \tabularnewline
        \hline
        \textbf{Salida esperada} & \texpected \tabularnewline
        \hline
      \end{tabularx}
      #2
    \end{table}
}
% end of command definition

% TABLA CASO DE PRUEBA

% define the key (arguments)
\makeatletter
\define@key{tckeys}{id}{%
  \def\tcid{#1}
}
\define@key{tckeys}{description}{%
  \def\tcdesc{#1}
}
\define@key{tckeys}{validates}{%
  \def\tcvalidates{#1}
}
\define@key{tckeys}{environment}{%
  \def\tcenv{#1}
}
\define@key{tckeys}{input}{%
  \def\tcinput{#1}
}
\define@key{tckeys}{output}{%
  \def\tcoutput{#1}
}
\makeatother
% end of key definition

% new command
\newcommand{\TestCase}[2][]{%
    \setkeys{tckeys}{#1}%
    \begin{table} [H]
      \begin{tabularx}{\linewidth}{|p{3cm}|X|}
        \hline
        \textbf{Caso de prueba} & \tcid \tabularnewline
        \hline
        \textbf{Descripción} & \tcdesc \tabularnewline
        \hline
        \textbf{Clases que valida} & \tcvalidates \tabularnewline
        \hline
        \textbf{Necesidades del entorno} & \tcenv \tabularnewline
        \hline
        \textbf{Entrada} & \tcinput \tabularnewline
        \hline
        \textbf{Salida esperada} & \tcoutput \tabularnewline
        \hline
      \end{tabularx}
      #2
    \end{table}
}
% end of command definition

% TABLA RIESGO

% define the key (arguments)
\makeatletter
\define@key{rkkeys}{id}{%
  \def\rkid{#1}
}
\define@key{rkkeys}{name}{%
  \def\rkname{#1}
}
\define@key{rkkeys}{description}{%
  \def\rkdesc{#1}
}
\define@key{rkkeys}{probability}{%
  \def\rkprobability{#1}
}
\define@key{rkkeys}{impact}{%
  \def\rkimpact{#1}
}
\define@key{rkkeys}{affected}{%
  \def\rkaffected{#1}
}
\define@key{rkkeys}{treatment}{%
  \def\rktreatment{#1}
}
\define@key{rkkeys}{indicators}{%
  \def\rkindicators{#1}
}
\define@key{rkkeys}{follow}{%
  \def\rkfollow{#1}
}
\makeatother
% end of key definition

% new command
\newcommand{\RiskItem}[2][]{%
    \setkeys{rkkeys}{#1}%
    \begin{table} [H]
      \begin{tabularx}{\linewidth}{|p{3cm}|X|}
        \hline
        \textbf{Riesgo \rkid} & \rkname \tabularnewline
        \hline
        \textbf{Descripción} & \rkdesc \tabularnewline
        \hline
        \textbf{Activos afectados} & \rkaffected \tabularnewline
        \hline
        \textbf{Probabilidad} & \rkprobability \tabularnewline
        \hline
        \textbf{Impacto} & \rkimpact \tabularnewline
        \hline
        \textbf{Tratamiento} & \rktreatment \tabularnewline
        \hline
        \textbf{Indicadores} & \rkindicators \tabularnewline
        \hline
        \textbf{Seguimiento} & \rkfollow \tabularnewline
        \hline
      \end{tabularx}
      #2
    \end{table}
}
% end of command definition